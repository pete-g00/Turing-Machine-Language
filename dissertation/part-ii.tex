\chapter{Turing Machine Parser and Website}

% What did you do to implement this idea, and what technical achievements did you make?
% \section{Guidance}
% You can't talk about everything. Cover the high level first, then cover important, relevant or impressive details.

% \section{General guidance for technical writing}

% These points apply to the whole dissertation, not just this chapter.

% \subsection{Figures}
% \emph{Always} refer to figures included, like Figure \ref{fig:relu}, in the body of the text. Include full, explanatory captions and make sure the figures look good on the page.
% You may include multiple figures in one float, as in Figure \ref{fig:synthetic}, using \texttt{subcaption}, which is enabled in the template.


% Figures are important. Use them well.
% \begin{figure}[htb]
%     \centering
%     \includegraphics[width=0.5\linewidth]{images/relu.pdf}    

%     \caption{In figure captions, explain what the reader is looking at: ``A schematic of the rectifying linear unit, where $a$ is the output amplitude,
%     $d$ is a configurable dead-zone, and $Z_j$ is the input signal'', as well as why the reader is looking at this: 
%     ``It is notable that there is no activation \emph{at all} below 0, which explains our initial results.'' 
%     \textbf{Use vector image formats (.pdf) where possible}. Size figures appropriately, and do not make them over-large or too small to read.
%     }

%     % use the notation fig:name to cross reference a figure
%     \label{fig:relu} 
% \end{figure}


% \begin{figure}[htb] 
%     \centering
%     \begin{subfigure}[b]{0.45\textwidth}
%         \includegraphics[width=\textwidth]{images/synthetic.png}
%         \caption{Synthetic image, black on white.}
%         \label{fig:syn1}
%     \end{subfigure}
%     ~ %add desired spacing between images, e. g. ~, \quad, \qquad, \hfill etc. 
%       %(or a blank line to force the subfigure onto a new line)
%     \begin{subfigure}[b]{0.45\textwidth}
%         \includegraphics[width=\textwidth]{images/synthetic_2.png}
%         \caption{Synthetic image, white on black.}
%         \label{fig:syn2}
%     \end{subfigure}
%     ~ %add desired spacing between images, e. g. ~, \quad, \qquad, \hfill etc. 
%     %(or a blank line to force the subfigure onto a new line)    
%     \caption{Synthetic test images for edge detection algorithms. \subref{fig:syn1} shows various gray levels that require an adaptive algorithm. \subref{fig:syn2}
%     shows more challenging edge detection tests that have crossing lines. Fusing these into full segments typically requires algorithms like the Hough transform.
%     This is an example of using subfigures, with \texttt{subref}s in the caption.
%     }\label{fig:synthetic}
% \end{figure}

% \clearpage

% \subsection{Equations}

% Equations should be typeset correctly and precisely. Make sure you get parenthesis sizing correct, and punctuate equations correctly 
% (the comma is important and goes \textit{inside} the equation block). Explain any symbols used clearly if not defined earlier. 

% For example, we might define:
% \begin{equation}
%     \hat{f}(\xi) = \frac{1}{2}\left[ \int_{-\infty}^{\infty} f(x) e^{2\pi i x \xi} \right],
% \end{equation}    
% where $\hat{f}(\xi)$ is the Fourier transform of the time domain signal $f(x)$.

% \subsection{Algorithms}
% Algorithms can be set using \texttt{algorithm2e}, as in Algorithm \ref{alg:metropolis}.

% % NOTE: line ends are denoted by \; in algorithm2e
% \begin{algorithm}
%     \DontPrintSemicolon
%     \KwData{$f_X(x)$, a probability density function returing the density at $x$.\; $\sigma$ a standard deviation specifying the spread of the proposal distribution.\;
%     $x_0$, an initial starting condition.}
%     \KwResult{$s=[x_1, x_2, \dots, x_n]$, $n$ samples approximately drawn from a distribution with PDF $f_X(x)$.}
%     \Begin{
%         $s \longleftarrow []$\;
%         $p \longleftarrow f_X(x)$\;
%         $i \longleftarrow 0$\;
%         \While{$i < n$}
%         {
%             $x^\prime \longleftarrow \mathcal{N}(x, \sigma^2)$\;
%             $p^\prime \longleftarrow f_X(x^\prime)$\;
%             $a \longleftarrow \frac{p^\prime}{p}$\;
%             $r \longleftarrow U(0,1)$\;
%             \If{$r<a$}
%             {
%                 $x \longleftarrow x^\prime$\;
%                 $p \longleftarrow f_X(x)$\;
%                 $i \longleftarrow i+1$\;
%                 append $x$ to $s$\;
%             }
%         }
%     }
    
% \caption{The Metropolis-Hastings MCMC algorithm for drawing samples from arbitrary probability distributions, 
% specialised for normal proposal distributions $q(x^\prime|x) = \mathcal{N}(x, \sigma^2)$. The symmetry of the normal distribution means the acceptance rule takes the simplified form.}\label{alg:metropolis}
% \end{algorithm}

% \subsection{Tables}

% If you need to include tables, like Table \ref{tab:operators}, use a tool like https://www.tablesgenerator.com/ to generate the table as it is
% extremely tedious otherwise. 

% \begin{table}[]
%     \caption{The standard table of operators in Python, along with their functional equivalents from the \texttt{operator} package. Note that table
%     captions go above the table, not below. Do not add additional rules/lines to tables. }\label{tab:operators}
%     %\tt 
%     \rowcolors{2}{}{gray!3}
%     \begin{tabular}{@{}lll@{}}
%     %\toprule
%     \textbf{Operation}    & \textbf{Syntax}                & \textbf{Function}                            \\ %\midrule % optional rule for header
%     Addition              & \texttt{a + b}                          & \texttt{add(a, b)}                                    \\
%     Concatenation         & \texttt{seq1 + seq2}                    & \texttt{concat(seq1, seq2)}                           \\
%     Containment Test      & \texttt{obj in seq}                     & \texttt{contains(seq, obj)}                           \\
%     Division              & \texttt{a / b}                          & \texttt{div(a, b) }  \\
%     Division              & \texttt{a / b}                          & \texttt{truediv(a, b) } \\
%     Division              & \texttt{a // b}                         & \texttt{floordiv(a, b)}                               \\
%     Bitwise And           & \texttt{a \& b}                         & \texttt{and\_(a, b)}                                  \\
%     Bitwise Exclusive Or  & \texttt{a \textasciicircum b}           & \texttt{xor(a, b)}                                    \\
%     Bitwise Inversion     & \texttt{$\sim$a}                        & \texttt{invert(a)}                                    \\
%     Bitwise Or            & \texttt{a | b}                          & \texttt{or\_(a, b)}                                   \\
%     Exponentiation        & \texttt{a ** b}                         & \texttt{pow(a, b)}                                    \\
%     Identity              & \texttt{a is b}                         & \texttt{is\_(a, b)}                                   \\
%     Identity              & \texttt{a is not b}                     & \texttt{is\_not(a, b)}                                \\
%     Indexed Assignment    & \texttt{obj{[}k{]} = v}                 & \texttt{setitem(obj, k, v)}                           \\
%     Indexed Deletion      & \texttt{del obj{[}k{]}}                 & \texttt{delitem(obj, k)}                              \\
%     Indexing              & \texttt{obj{[}k{]}}                     & \texttt{getitem(obj, k)}                              \\
%     Left Shift            & \texttt{a \textless{}\textless b}       & \texttt{lshift(a, b)}                                 \\
%     Modulo                & \texttt{a \% b}                         & \texttt{mod(a, b)}                                    \\
%     Multiplication        & \texttt{a * b}                          & \texttt{mul(a, b)}                                    \\
%     Negation (Arithmetic) & \texttt{- a}                            & \texttt{neg(a)}                                       \\
%     Negation (Logical)    & \texttt{not a}                          & \texttt{not\_(a)}                                     \\
%     Positive              & \texttt{+ a}                            & \texttt{pos(a)}                                       \\
%     Right Shift           & \texttt{a \textgreater{}\textgreater b} & \texttt{rshift(a, b)}                                 \\
%     Sequence Repetition   & \texttt{seq * i}                        & \texttt{repeat(seq, i)}                               \\
%     Slice Assignment      & \texttt{seq{[}i:j{]} = values}          & \texttt{setitem(seq, slice(i, j), values)}            \\
%     Slice Deletion        & \texttt{del seq{[}i:j{]}}               & \texttt{delitem(seq, slice(i, j))}                    \\
%     Slicing               & \texttt{seq{[}i:j{]}}                   & \texttt{getitem(seq, slice(i, j))}                    \\
%     String Formatting     & \texttt{s \% obj}                       & \texttt{mod(s, obj)}                                  \\
%     Subtraction           & \texttt{a - b}                          & \texttt{sub(a, b)}                                    \\
%     Truth Test            & \texttt{obj}                            & \texttt{truth(obj)}                                   \\
%     Ordering              & \texttt{a \textless b}                  & \texttt{lt(a, b)}                                     \\
%     Ordering              & \texttt{a \textless{}= b}               & \texttt{le(a, b)}                                     \\
%     % \bottomrule
%     \end{tabular}
%     \end{table}
% \subsection{Code}

% Avoid putting large blocks of code in the report (more than a page in one block, for example). Use syntax highlighting if possible, as in Listing \ref{lst:callahan}.

% \begin{lstlisting}[language=python, float, caption={The algorithm for packing the $3\times 3$ outer-totalistic binary CA successor rule into a 
%     $16\times 16\times 16\times 16$ 4 bit lookup table, running an equivalent, notionally 16-state $2\times 2$ CA.}, label=lst:callahan]
%     def create_callahan_table(rule="b3s23"):
%         """Generate the lookup table for the cells."""        
%         s_table = np.zeros((16, 16, 16, 16), dtype=np.uint8)
%         birth, survive = parse_rule(rule)

%         # generate all 16 bit strings
%         for iv in range(65536):
%             bv = [(iv >> z) & 1 for z in range(16)]
%             a, b, c, d, e, f, g, h, i, j, k, l, m, n, o, p = bv

%             # compute next state of the inner 2x2
%             nw = apply_rule(f, a, b, c, e, g, i, j, k)
%             ne = apply_rule(g, b, c, d, f, h, j, k, l)
%             sw = apply_rule(j, e, f, g, i, k, m, n, o)
%             se = apply_rule(k, f, g, h, j, l, n, o, p)

%             # compute the index of this 4x4
%             nw_code = a | (b << 1) | (e << 2) | (f << 3)
%             ne_code = c | (d << 1) | (g << 2) | (h << 3)
%             sw_code = i | (j << 1) | (m << 2) | (n << 3)
%             se_code = k | (l << 1) | (o << 2) | (p << 3)

%             # compute the state for the 2x2
%             next_code = nw | (ne << 1) | (sw << 2) | (se << 3)

%             # get the 4x4 index, and write into the table
%             s_table[nw_code, ne_code, sw_code, se_code] = next_code

%         return s_table

% \end{lstlisting}

\section{Parser}
\subsection{Design}

\subsection{Implementation}


\section{Website}
\subsection{Design}

\subsection{Implementation}