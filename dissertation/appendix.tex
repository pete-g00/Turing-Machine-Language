
\begin{appendices}

% Use separate appendix chapters for groups of ancillary material that support your dissertation. 
% Typical inclusions in the appendices are:

% \begin{itemize}
% \item
%   Copies of ethics approvals (you must include these if you needed to get them)
% \item
%   Copies of questionnaires etc. used to gather data from subjects. Don't include
%   voluminous data logs; instead submit these electronically alongside your source code.
% \item
%   Extensive tables or figures that are too bulky to fit in the main body of
%   the report, particularly ones that are repetitive and summarised in the body.
% \item Outline of the source code (e.g. directory structure), 
%     or other architecture documentation like class diagrams.
% \item User manuals, and any guides to starting/running the software. 
% Your equivalent of \texttt{readme.md} should be included.

% \end{itemize}

% \textbf{Don't include your source code in the appendices}. It will be
% submitted separately.

\chapter{TML EBNF}
The following is the EBNF for the Turing Machine Language.
\begin{align*}
    \textit{program} &= \textit{alphabet} \ \textit{module}^+ \\
    \textit{alphabet} &= \texttt{alphabet} \ \texttt{=} \ \texttt{\{} \ \textit{seq-val} \ \texttt{\}} \\
    \textit{module} &= \texttt{module} \ \textit{id} \ \texttt{\{} \ \textit{block}^+ \ \texttt{\}} \\
    \textit{block} &= \textit{basic-block} \ | \ \textit{switch-block} \\
    \textit{switch-block} &= \texttt{switch tapehead \{} \ \textit{case-block}^+ \ \texttt{\}} \\
    \textit{case-block} &= \textit{if-block} \ | \ \textit{while-block} \\
    \textit{if-block} &= \texttt{if} \ \textit{seq-val} \ \texttt{\{} \textit{block}^+ \texttt{\}} \\
    \textit{while-block} &= \texttt{while} \ \textit{seq-val} \ \texttt{\{} \ \textit{core-com}^+ \ \texttt{\}} \\
    \textit{basic-block} &= (\textit{core-com} \ | \ \textit{flow-com})^+ \\
    \textit{core-com} &= \texttt{move} \ \textit{direction} \ | \ \texttt{changeto} \ \textit{value} \\
    \textit{flow-com} &= \texttt{goto} \ \textit{id} \ | \ \textit{terminate} \\
    \textit{terminate} &= \texttt{reject} \ | \ \texttt{accept} \\
    \textit{direction} &= \texttt{left} \ | \ \texttt{right} \\
    \textit{seq-val} &= (\textit{value} \texttt{,})^* \ \textit{value} \\
    \textit{value} &= \texttt{blank} \ | \ \texttt{a} \ | \ \texttt{b} \ | \ \texttt{c} \ | \ \dots \ | \ \texttt{z} \ | \ \texttt{0} \ | \ \texttt{1} \ | \ \dots \ | \ \texttt{9} \\
    \textit{id} &= (\texttt{a} \ | \ \texttt{b} \ | \ \texttt{c} \ | \ \dots \ | \ \texttt{z} \ | \ \texttt{A} \ | \ \texttt{B} \ | \ \texttt{C} \ | \ \dots \ | \ \texttt{Z})^+
\end{align*}

\chapter{Proofs of the Theorems}
\setcounter{theorem}{0}

\begin{theorem} \label{thm:complete_TM}
    Let $P$ be a valid TML program. Then, $P$ and its completion $P^+$ execute on every valid tape $T$ in the same way. That is,
    \begin{itemize}
        \item for every valid index $n$, if we have tape $T_n$, tapehead index $i_n$ and module $m_n$ with executing block $b_n$ for the TM program $P$, and we have tape $S_n$, tapehead index $j_n$ and module $t_n$, then $T_n = S_n$, $i_n = j_n$, and $t_n$ is the corresponding complete module block of $b_n$;
        \item $P$ terminates execution on $T$ if and only if $P^+$ terminates execution on $T$, with the same final status (\texttt{accept} or \texttt{reject}).
    \end{itemize}
\end{theorem}
\begin{proof}
    We prove this by induction on the execution step (of the tape). 
    \begin{itemize}
        \item At the start, we have the same tape $T$ for both $P$ and $P^+$, with tapehead index 0. Moreover, the corresponding (completed) module of the first block in the first module of $P$ is the first module of $P$. So, the result is true if $n = 0$. 
        \item Now, assume that the result is true for some integer $n$, where the block $b_n$ in the TML program $P$ does not end with a terminating \textit{flow} command. Let $\sigma_n$ be the letter at index $i_n = j_n$ on the tape $S_n = T_n$.
        \begin{itemize}
            \item If the \textit{changeto} command is missing in $b_n$ for $\sigma_n$, then the next tape $T_{n+1} = T_n$. In the complete module $m_n$, the case for $\sigma_n$ will have the command \texttt{changeto} $\sigma_n$. So, the next tape is given by:
            \[S_{n+1}(x) = \begin{cases}
                S_n(x) & x \neq j_n \\
                \sigma_n & \text{otherwise}
            \end{cases}.\]
            Therefore, we have $S_{n+1} = S_n$ as well. So, $T_{n+1} = S_{n+1}$. Otherwise, we have the same \textit{changeto} command in the two blocks, in which case $T_{n+1} = S_{n+1}$ as well.
            
            \item If the \textit{move} command is missing in $b_n$ for $\sigma_n$, then the next tapehead index $i_{n+1} = i_n - 1$. In the complete module $m_n$, the case for $\sigma_n$ will have the command \texttt{move left}, so we also have $j_{n+1} = j_n - 1$. Applying the inductive hypothesis, we have $i_{n+1} = j_{n+1}$. Otherwise, we have the same \textit{move} command, meaning that $i_{n+1} = j_{n+1}$ as well.
            
            \item We now consider the next block $b_{n+1}$:
            \begin{itemize}
                \item If the block $b_n$ is a \textit{switch} block with a \textit{while} case for $\sigma_n$, then this is still true in the module $m_n$. So, the next block to be executed in $P$ is $b_n$, and the next module to be executed in $P^+$ is $m_n$. In that case, the corresponding module of the block $b_{n+1} = b_n$ is still $m_{n+1} = m_n$.
    
                \item Instead, if the block $b_n$ has no \textit{flow} command for $\sigma_n$, and is not the last block, then the next block to execute is the block just below $b_n$, referred as $b_{n+1}$. By the definition of $P^+$, we find that the case block in the module $m_n$ has a \textit{goto} command, going to the module $m_{n+1}$ which corresponds to the block $b_{n+1}$. 
            
                \item Now, if the \textit{flow} command is missing for $\sigma_n$ and this is the last block, then execution is terminated with the status \texttt{reject} for the program $P$. In that case, the case for $\sigma_n$ in the module $m_n$ has the \texttt{reject} command present, so the same happens for $P^+$ as well. 
            
                \item Otherwise, both $P$ and $P^+$ have the same flow command, meaning that there is either correspondence between the next module to be executed, or both the program terminate with the same status. 
            \end{itemize}
        \end{itemize}
    \end{itemize}
    In that case, $P$ and $P^+$ execute on $T$ the same way by induction.
\end{proof}

\begin{theorem} \label{thm:TM_to_TMP}
    Let $M$ be a TM, and let $P$ be the corresponding program for $M$. Then, $M$ and $P$ execute on every valid tape $T$ in the same way. That is, 
    \begin{itemize}
        \item for every valid index $n$, if we have tape $T_n$, tapehead index $i_n$ and module $m_n$ for the TM program $P$, and we have tape $S_n$, tapehead index $j_n$ and state $q_n$ for the TM $M$, then $T_n = S_n$, $i_n = j_n$ and $m_n$ is the corresponding module for $q_n$;
        \item $M$ terminates execution on $T$ if and only if $P$ terminates execution on $T$, with the same final status (\texttt{accept} or \texttt{reject}).
    \end{itemize}
\end{theorem}
\begin{proof}
    We prove this by induction on the execution step. 
    \begin{itemize}
        \item At the start, we have the same tape $T$ for both $M$ and $P$, with tapehead index $0$. Moreover, the first module in $P$ corresponds to the initial state $q_0$. So, the result is true if $n = 0$.
        
        \item Now, assume that the result is true for some integer $n$, where the TM state $q_n$ is not \texttt{accept} or \texttt{reject}. In that case, $T_n = S_n$, $i_n = j_n$ and $m_n$ is the corresponding module for $q_n$. Let $\sigma_n$ be the letter at index $i_n = j_n$ on the tape $T_n = S_n$. Denote $q(q_n, \sigma_n) = (q_{n+1}, \sigma_{n+1}, \texttt{dir})$. In that case,
        \[T_{n+1}(x) = \begin{cases}
            T_n(x) & x \neq i_n \\
            \sigma_{n+1} & \text{otherwise},
        \end{cases} \qquad i_{n+1} = \begin{cases}
            i_n - 1 & \texttt{dir} = \texttt{left} \\
            i_n + 1 & \texttt{dir} = \texttt{right},
        \end{cases}\]
        and the next state is $q_{n+1}$. 
        
        \begin{itemize}
            \item We know that the module $m_n$ in TM program $P$ corresponds to the state $q_n$, so it has a \texttt{changeto} $\sigma_{n+1}$ command for the case $\sigma_n$. In the case, the next tape for $P$ is:
            \[S_{n+1}(x) = \begin{cases}
                S_n(x) & x \neq i_n \\
                \sigma_{n+1} & \text{otherwise}.
            \end{cases}\]
            So, $T_{n+1} = S_{n+1}$. 
            
            \item Similarly, the case also contains a \texttt{move dir} command. This implies that the next tapehead index for $P$ is:
            \[j_{n+1} = \begin{cases}
                j_n - 1 & \texttt{dir} = \texttt{left} \\
                j_n + 1 & \texttt{dir} = \texttt{right}.
            \end{cases}\]
            Hence, $i_{n+1} = j_{n+1}$. 
        
            \item Next, we consider the value of $q_{n+1}$:
            \begin{itemize}
                \item If $q_{n+1} = q_n$, then the case block is a \textit{while} block, and vice versa. So, the next module to be executed is $m_n$. In that case, $m_{n+1}$ still corresponds to $q_{n+1}$.
                \item Otherwise, we have an \textit{if} block. 
                \begin{itemize}
                    \item In particular, if $q_{n+1}$ is the \texttt{accept} state, then the case for $\sigma_n$ contains the \textit{flow} command \texttt{accept}, and vice versa. In that case, execution terminates with the same final status of \texttt{accept}. The same is true for \texttt{reject}. 
                    \item Otherwise, the module contains the command \texttt{goto} $m_{n+1}$, where $m_{n+1}$ is the corresponding module for $q_{n+1}$.
                \end{itemize}
            \end{itemize}
        \end{itemize}
        % Therefore, if the result holds for $n$, it holds for $n+1$. So, the result follows from induction.
    \end{itemize}
    In that case, $P$ and $M$ execute on $T$ the same way by induction.
\end{proof}

\begin{theorem}
    Let $P$ be a complete TM program, and let $M$ be the corresponding TM for $P$. Then, $P$ and $M$ execute on every valid tape $T$ in the same way. That is,
    \begin{itemize}
        \item for every valid index $n$, if we have tape $T_n$, tapehead index $i_n$ and module $m_n$ for TM program $P$, and we have tape $S_n$, tapehead index $j_n$ and state $q_n$ for the TM $M$, then $T_n = S_n$, $i_n = j_n$ and $q_n$ is the corresponding state for $m_n$;
        \item $P$ terminates execution on $T$ if and only if $M$ terminates execution on $T$, with the same final status (\texttt{accept} or \texttt{reject}).
    \end{itemize}
\end{theorem}
\begin{proof}
    We prove this as well by induction on the execution step of the tape. 
    \begin{itemize}
        \item At the start, we have the same tape $T$ for both $P$ and $M$, with tapehead index $0$. Moreover, the initial state $q_0$ in $M$ corresponds to the first module in $P$. So, the result is true if $n = 0$. 
        
        \item Now, assume that the result is true for some integer $n$, which is not the terminating step in execution. In that case, $S_n = T_n$, $j_n = i_n$ and $q_n$ is the corresponding state for $m_n$. Let $\sigma_n$ be the letter at index $j_n = i_n$ on the tape $S_n = T_n$. We now consider the single switch block in $m_n$:
        \begin{itemize}
            \item If the block in $m_n$ corresponding to $\sigma_n$ is a \textit{while} block, then we know that its body is partially complete, and so is composed of the following commands:
            \begin{itemize}
                \item \texttt{changeto} $\sigma_{n+1}$
                \item \texttt{move dir}
            \end{itemize}
            So, we have $\delta(q_n, \sigma_n) = (q_n, \sigma_{n+1}, \texttt{dir})$. Using the same argument as in Theorem \ref{thm:TM_to_TMP}, we find that $T_{n+1} = S_{n+1}$ and $i_{n+1} = j_{n+1}$. Also, $q_{n+1} = q_n$ is the corresponding state for $m_{n+1} = m_n$. 
            
            \item Otherwise, we have an \textit{if} command. In this case, the case body is complete, and so composed of the following commands:
            \begin{itemize}
                \item \texttt{changeto} $\sigma_{n+1}$
                \item \texttt{move dir}
                \item \texttt{accept}, \texttt{reject} or \texttt{goto} $m_{n+1}$.
            \end{itemize}
            So, we have $\delta(q_n, \sigma_n) = (q_{n+1}, \sigma_{n+1}, \texttt{dir})$, where $q_{n+1}$ is the corresponding state to the \textit{flow} command present. Here too, we have $T_{n+1} = S_{n+1}$ and $i_{n+1} = j_{n+1}$ by construction. 
        \end{itemize}
        Now, we consider the flow command:
        \begin{itemize}
            \item If we have an \texttt{accept} command in the body, then $q_{n+1}$ is the accepting state, and vice versa. So, we terminate execution with the final status of \texttt{accept}. The same is true for \texttt{reject}. 
            \item Otherwise, the state $q_{n+1}$ is the corresponding state to the module $m_{n+1}$.
        \end{itemize}
        In all cases, there is a correspondence between the state for $m_{n+1}$ and $q_{n+1}$.   
    \end{itemize}
    So, the result follows from induction.
\end{proof}



\end{appendices}
