\chapter{Introduction}

\pagenumbering{arabic} 

\section{Motivation}
When students are asked to create an algorithm using Turing Machines, they typically do so in two steps:
\begin{enumerate}
    \item they plan the algorithm to understand the evolution of the tape during execution; and then
    \item they convert this algorithm into an actual Turing Machine.
\end{enumerate}
Since students are expected to be pretty familiar with coding, the first step could be done programmatically!

This project proposes a programming language for Turing Machines that allows students to plan tape execution rigorously (step 1) while abstracting the Turing Machine operations (step 2). It is hoped that using the language helps student better construct Turing Machines.

\section{Objectives}
There are 3 parts to the project: defining the language; creating the parser for the language; and creating a website that allows a user to make use of the parser.

The first aim is to define a programming language to represent Turing machines, called the Turing Machine Language. The language will allow students to devise an algorithm to execute on tape while abstracting the definition of a Turing Machine.

Next, a parser is to be created for the language. This parser should be able to take in a string representation of a program, and then parse it into a program context. Then, a program context can be 
\begin{itemize}
    \item validated to ensure it has no errors; 
    \item converted into a Turing machine; and
    \item executed on a tape.
\end{itemize}

Finally, a website is to be created to allow the user to access the parser. It should feature an editor to allow users to type a program. The user would then be able to convert it into a Turing machine, or execute it on a valid tape.

\section{Summary}
\begin{itemize}
    \item \textbf{Chapter 2} contains background information on Turing Machines and the parsing process;
    \item \textbf{Chapter 3} lists the requirements for the project;
    \item \textbf{Chapter 4} illustrates the design of the language, the parser and the product;
    \item \textbf{Chapter 5} demonstrates the implementation of the parser and the product;
    \item \textbf{Chapter 6} outlines the results of the evaluation, along with some limitations to the process; and
    \item \textbf{Chapter 7} concludes the dissertation with a summary and highlights some recommendations for future work.
\end{itemize}
