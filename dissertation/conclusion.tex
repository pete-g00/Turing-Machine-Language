\chapter{Conclusion}

% Summarise the whole project for a lazy reader who didn't read the rest (e.g. a prize-awarding committee). This chapter should be short in most dissertations; maybe one to three pages.
% \section{Guidance}
% \begin{itemize}
%     \item
%         Summarise briefly and fairly.
%     \item
%         You should be addressing the general problem you introduced in the
%         Introduction.        
%     \item
%         Include summary of concrete results (``the new compiler ran 2x
%         faster'')
%     \item
%         Indicate what future work could be done, but remember: \textbf{you
%         won't get credit for things you haven't done}.
% \end{itemize}

% \section{Summary}
% Summarise what you did; answer the general questions you asked in the introduction. What did you achieve? Briefly describe what was built and summarise the evaluation results.

% \section{Reflection}
% Discuss what went well and what didn't and how you would do things differently if you did this project again.

\section{Future Work}

There are many additions that can be made to the product in future. Some of these were discovered during production, and some as part of user evaluation.

\subsection{Language}
Although the language is equivalent to TMs, it is still quite low-level. There are many common paradigms that can be added to the language. These include:
\begin{enumerate}
    \item the ability to traverse to the end and start in one command, e.g. \texttt{move start} and \texttt{move end};
    \item an else clause (an if block for the remaining characters); 
    % \item the ability to shift all the characters by one blank space to left or right, e.g. \texttt{shift left} and \texttt{shift right}; and
    \item the ability for modules to be parametrised with respect to letters.
\end{enumerate}
The first feature is a very common feature seen in plenty of programs. It was brought up plenty of times during the evaluation sessions and given as an example of a feature to add in the survey. The second point is also somewhat common and was suggested by a student. These two features should not be difficult to add to the language. 

The final feature was also discovered during the survey. Students were quite frustrated when they realised that the programs they had written would not work since it would introduce a blank entry within the middle of the tape. This feature would certainly remedy this issue, but adding this feature might make TML much simpler than TMs- students are \emph{expected} to write programs in a way that ensures that no spaces are added to the tape. Nonetheless, it is possible to write a module that shifts values to the left or the right, so if implemented, this can be added as an in-built module.

% The final feature is quite interesting. There are many programs where some modules are very similar and only differ in a character. For instance, in the \texttt{palindrome} program with alphabet $\{\texttt{a}, \texttt{b}\}$, the case where the first letter is \texttt{a} or \texttt{b} are almost identical, but only differ in the letters to accept or reject. Hence, this would be a great feature to add, and would get rid of many duplicate/similar code within programs! This would also serve as a benefit to TMs since the ability to parametrise is not possible there.

Another feature that could be added to the language is the ability to convert a TM into a `good' TML program. Currently, the proof of equivalence only provides a way to convert TMs into complete TML programs. Moreover, the proof relies on a one-way conversion from valid TML programs to complete TML programs. Hence, it would be a good idea to consider converting a TM (or equivalent, a complete TML program) to a TML program that is nested, and not complete. One way to do so would be to achieve maximum nesting by replacing \texttt{goto} commands with the actual module, however this is not optimal. In particular, we would have a lot of duplicate code present. Instead, it might be a good idea to only replace modules that are only called once, and to keep modules that are called multiple times separate. 

 \subsection{Website}
There are many possible improvements to the website, which were discovered during production and user evaluation. These were some of the features that can be added:
\begin{itemize}
    \item support for direct execution of a TM;
    \item a play button on the tape section to execute long programs on long tapes without pressing the step button hundreds of times;
    \item the ability to collapse the panels; and
    \item the ability to customise the number of tape entries to be shown.
\end{itemize}
All of these are great features that could be added to the website and would make it more usable and easier to keep track of code execution. 

Another feature that can be added is to make the website more responsive. The FSM is currently being produced using the graphviz framework. The resulting image has hard-coded dimensions which makes it hard to make the website responsive. It is not completely possible to make the graph fixed; only the maximum size can be specified. Moreover, for complex FSMs with 20 nodes, the nodes are quite small so it is hard to follow code. To fix this, one of the following features could be added:
\begin{itemize}
    \item the ability to zoom in and drag the FSM panel; or
    \item the ability to scroll the FSM panel.
\end{itemize}

