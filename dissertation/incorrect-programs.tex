\chapter{Incorrect TML Programs}

Below is a selection of incorrect TML programs, along with an explanation as to why they are incorrect.
\begin{itemize}
    \item \textbf{Invalid while block (flow command)}
\begin{lstlisting}[language=TML]
alphabet = {a, b}
module main {
    while a, b {
        accept
    }
}
\end{lstlisting}
    A while block cannot have a flow command. We know that the next block to be executed is the same block, so this doesn't make sense- are we meant to terminate or re-run the current block?

    \item \textbf{Invalid while block (multiple blocks)}
\begin{lstlisting}[language=TML]
alphabet = {a, b}
module main {
    while blank {
        move left
        move right
    }
}
\end{lstlisting}
    A while block must be composed of a single basic block. We know that a while block corresponds to a self-loop, so if we have 2 blocks, we would need another state to link to the current state. This doesn't make sense, so we can only have one block. The block must also be a simple block since we have already fixed what character the while block applies to.

    \item \textbf{Invalid switch block (first block not basic)}
\begin{lstlisting}[language=TML]
alphabet = {a, b}
module main {
    if a, blank {
        if a {
            move right
        }
    }
}
\end{lstlisting}
    The first block in a switch block must be a basic block. That is, we cannot have nested if blocks without an intermediate command. This does not make sense semantically- in the case above, we should just have a single if block for \texttt{a}.

    \item \textbf{Invalid module (flow command not final)}
\begin{lstlisting}[language=TML]
alphabet = {a, b}
module main {
    goto simple
    move left
}
module simple {
    reject
}
\end{lstlisting}
    If we have a \emph{flow}-command in a block, then it must be the final command. A \emph{flow}-command moves to terminating execution or executing a module from the start, so any command below it cannot be executed. In this case, we are moving to a different block called \texttt{simple}, so we never move \texttt{left}.

    \item \textbf{Invalid module (switch block not final)}
\begin{lstlisting}[language=TML]
alphabet = {a, b}
module a {
    if a, b {
        changeto blank
        move right
        move right
        changeto b
    } if blank {
        move left
        changeto a
        reject
    }
    accept
}
\end{lstlisting}
    If a module has a \textit{switch} block, it must be the final block. In this case, that is not the case. It is possible that the program is meant to accept if we execute the \textit{if}-block at line 3, but it is not possible to continue the \textit{if}-block at line 8- this block already ends with a flow command. So, the execution would be unclear if this was allowed.

    \item \textbf{Invalid module (invalid name)}
\begin{lstlisting}[language=TML]
alphabet = {a, b}
module accept {
    move right
}
\end{lstlisting}
    A module cannot be named \texttt{accept} or \texttt{reject}. It can be named any of the other keywords, but not \texttt{accept} or \texttt{reject}.

\end{itemize}
